%XeLaTeX
\documentclass[14pt, oneside]{altsu-report}

\worktype{Отчёт по практике на тему:}
\title{РАЗРАБОТКА КРОССПЛАТФОРМЕННОГО ИГРОВОГО ПРИЛОЖЕНИЯ FLAPPY BIRD ДЛЯ LINUX и WINDOWS С ГРАФИЧЕСКИМ ИНТЕРФЕЙСОМ}
\author{В.\,Г.~Иванова}
\groupnumber{5.205-1}
\GradebookNumber{1337}
\supervisor{И.\,А.~Шмаков}
\supervisordegree{к.ф.-м.н., доцент}
\ministry{Министерство науки и высшего образования}
\country{Российской Федерации}
\fulluniversityname{ФГБОУ ВО Алтайский государственный университет}
\institute{Институт цифровых технологий, электроники и физики}
\department{Кафедра вычислительной техники и электроники}
\departmentchief{В.\,В.~Пашнев}
\departmentchiefdegree{к.ф.-м.н., доцент}
\shortdepartment{ВТиЭ}
\abstractRU{Данная работа посвящена разработке кроссплатформенного игрового приложения в Unity. Целью является создание игры «Flappy Bird» с графическим интерфейсом для таких операционных систем как Linux и Windows на языке C\#. Для выполнения поставленной задачи необходимо было изучить литературу по данной теме. 

В заключение следует отметить, что цели, поставленные в начале данной работы, были достигнуты. Данный проект является актуальным, поскольку в наши дни разработка различных проектов становится все более популярной и не каждый из них получается доработать и внедрить на различные операционные системы.}

\abstractEN{This work is devoted to the development of a cross-platform game application in Unity. The goal is to create a "Flappy Bird" game with a graphical interface for operating systems such as Linux and Windows in C\#. To complete the task, it was necessary to study the literature on this topic. 

In conclusion, it should be noted that the goals set at the beginning of this work have been achieved. This project is relevant because nowadays the development of various projects is becoming more and more popular and not every one of them can be finalized and implemented on various operating systems.}

%\keysRU{компьютерное моделирование, cистема управления версиями}
%\keysEN{computer simulation, distributed version control}

\date{\the\year}

% Подключение файлов с библиотекой.
\addbibresource{graduate-students.bib}

% Пакет для отладки отступов.
%\usepackage{showframe}

\begin{document}
\maketitle

\setcounter{page}{2}
\makeabstract
\tableofcontents

\chapter*{Введение}
\phantomsection\addcontentsline{toc}{chapter}{ВВЕДЕНИЕ}

Индустрия компьютерных игр является одной из самой перспективной и быстроразвивающейся областью в мире информационных технологий. Игры находят в себе не только развлекательный характер, но и учебный, что способствует развитию различных качеств как у взрослых, так и у детей. Например, игры способны развивать внимание, общительность, формируют логическое мышление, расширяют кругозор. Существуют также разные категории игр, которые делают упор в большую степень развития качеств человека.

На сегодняшний день создание игровых приложений не ограничивает пользователей различных операционных систем, что позволяет разрабатывать игры, которые никак не будут конфликтовать между разными платформами. Благодаря использованию кросс-платформенных инструментов и технологий, разработчики могут создавать игровые приложения, которые могут быть запущены на разных устройствах, таких как персональные компьютеры, смартфоны, планшеты и игровые консоли.

В данной работе будет разработано приложение под  Linux и Windows под известным названием «Flappy Bird» с помощью кроссплатформенной среды Unity.

\textbf{Актуальность:} 
Актуальность данной работы заключается в воссоздании игры «Flappy Bird» с помощью кроссплатформенной среды Unity с дополнительными доработками интерфейса. 

\textbf{Цель:}
Целью данной работы является разработка кроссплатформенного игрового приложения «Flappy Bird» с графическим интерфейсом для таких операционных систем как Linux и Windows.

\textbf{Задачи:}
\begin{enumerate}
\item Раскрыть понятие кроссплатформенности.
\item Изучить кроссплатформенную среду разработки компьютерных игр Unity.
\item Изучить стандартную библиотеку С\#.
\item Изучить использование методов С\#.
\item Внедрить игровой интерфейс к игре.
\end{enumerate}

Благодаря данной работе можно будет на практике применять свои навыки программирования на C\# и использование среды Unity в дальнейшем, а также расширить имеющиеся знания и области применения данного языка и платформы, так как это актуально в современном мире.


\chapter{ГЛАВА 1. ТЕОРЕТИЧЕСКАЯ ЧАСТЬ} 

Создание игровых приложений стало более доступным и простым благодаря развитию интегрированных сред разработки (IDE) и готовых игровых движков, которые предоставляют широкий набор инструментов и ресурсов для разработки игр. Это позволяет как профессионалам, так и начинающим разработчикам воплощать свои идеи в жизнь и создавать качественные игровые продукты.

Одной из такие сред и является Unity - кроссплатформенная среда разработки игр. 

Кроссплатформенность - это способность программного обеспечения работать с несколькими операционными системами. 

Unity предлагает моделирование физических сред, карты нормалей, преграждение окружающего света в экранном пространстве, динамические тени и т.д. 

У среды также есть основные преимущества перед другими инструментами разработки игр:

\begin{enumerate}
\item Производительный визуальный рабочий процесс
\item Межплатформенная поддержка.
\end{enumerate} 

Unity поддерживает различные платформы, включая ПК, мобильные устройства, игровые консоли, виртуальную и дополненную реальность ~\cite{Unity, Unity2, Unity5}.

Платформа также предлагает широкий выбор ресурсов, что позволяет разработчикам использовать готовые модели, текстуры и другие элементы для ускорения процесса разработки игры. 

Данная среда использует язык программирования C\# в качестве основного языка разработки. C\# - это объектно-ориентированный язык программирования, разработанный Microsoft. Он предоставляет возможности для разработки игр и приложений, включая синтаксическую ясность, безопасность типов, а также удобные инструменты для управления памятью.

Unity обеспечивает интеграцию с языком C\# и предоставляет разработчикам широкий набор программирования, который позволяет взаимодействовать с различными компонентами и системами игрового движка. Разработчики могут использовать C\# для создания игровой логики, управления объектами и персонажами, работы с анимацией и многого другого. 

C\# в Unity имеет множество преимуществ, включая хорошую производительность, удобство разработки и интеграцию с другими инструментами среды. Платформа также предоставляет среду разработки Visual Studio и Visual Studio Code, которые обеспечивают инструменты для отладки с помощью скриптов ~\cite{Unity3, Unity4}. 

Благодаря своей гибкости и мощным возможностям, Unity стал одним из популярных инструментов для разработки игр и приложений в игровой индустрии.

В данной среде будет разрабатываться приложение под названием «Flappy Bird» — это игра на реакцию. В ней нужно управлять птичкой и пролететь через трубы.

Управление птичкой - для поддержания полета достаточно нажимать на левую кнопку мыши. Цель игры — пролететь возможное большее расстояние. Если птичка сталкивается с трубой, то игра завершается. После проигрыша можно начать игру заново.

\chapter{ГЛАВА 2. ПРАКТИЧЕСКАЯ ЧАСТЬ}

\textbf{Подготовка к разработке}

Разработка игры начинается с создания проекта в среде Unity. Он будет называться «Flappy Bird». Вся разработка приложения будет идти на версии 2021.3.4f1. Далее выбирается 2D шаблон разработки, так как сама игра «Flappy Bird» разработана под двухмерное пространство. 

Данный шаблон заранее включает различные 2D-пакеты для разработки таких игр. 

Например это: 

\begin{enumerate}
\item 2D-анимация.
\item Двумерные источники света.
\item Sprite Shape (создание окружения с помощью различных форм).
\item Tilemaps (создание шестиугольные и изометрические карты).
\end{enumerate} 

При создании игры понадобятся 2D-анимации и  Sprite Shape. Самое важное, что этот пакет разработан под платформы iOS, Android, Linux, macOS, WebGL, Windows, а значит он обладает нужной кроссплатформенностью под цели работы.

При загрузке рабочего пространства Unity выдаст нам что-то по типу "сцены", в которой и начнется разработка. Для начала добавим спрайты в папку «Kartinks», к части которых будут писаться в дальнейшем скрипты, проектироваться анимация и разрабатываться окружение с помощью различных фигур.

Спрайты - это картинки в 2D-играх, из которых состоят игровые персонажи, монстры, движущиеся объекты и т.д.

В данном проекте будет около n спрайтов. 

\begin{enumerate}
\item Bird01, Bird02, Bird03 - спрайты для птички. Из них будет создаваться анимация полета.
\item Background - задний фон, над которым можно будет поработать с помощью пакета Sprite Shape.
\item Pipe - трубы.
\item Ground - земля локации из которой будут исходить трубы.
\end{enumerate} 

После добавления спрайтов им можно изменить настройки импорта, что и сделается для всех картинок для более правильной композиции игры.

По умолчанию Pixels Per Unit = 100. 100 пикселей на единицу будет означать, что спрайт, который равен 100 пикселям, будет равен 1 единице сцены. Это расширение слишком большое для проекта, птичка на сцене будет очень маленькой, оптимальнее для нее будет поставить Pixels Per Unit = 24.

Format = RGBA 32 bit. Ставится для того, чтобы не было резкого перехода цветов.

Max Size = 256. Ставится из-за того, что размеры спрайтов не превышают данное значение.

Filter Mode = Point(no filter). Когда установлено значение «Point», текстуры не подвергаются фильтрации при масштабировании, что означает, что каждый пиксель исходной текстуры будет соответствовать одному пикселю на экране. Это используется для сохранения пиксельности этой игры.

После данных действий можно начать добавлять на сцену элементы игры и писать к ним скрипты.

\textbf{Программирование птички}

После добавления птички на сцену, ей нужно задать дополнительные компоненты. В разделе «Tag» мы выбираем «Player», это делается для того, чтобы создать некое единство игры с главным объектом, на который в последствии можно будет ссылаться. То есть, объекту птички мы просто присваиваем метку или идентификатор.

Добавим дополнительно компонент Rigidbody 2D. Он Unity является компонентом, который используется для моделирования физики движения 2D объектов. Коомпонент применяется к игровому объекту и позволяет симулировать движение, столкновения и взаимодействия сил в двумерной сцене. Здесь птичке нужно задать Bode Type = Kinematic. Когда установлено это значение, объекты с таким компонентом игнорируют физическую симуляцию и не подвергаются воздействию сил гравитации, столкновений или других физических эффектов.

Также добавим Circle Collaider 2D. Это компонент в Unity, который используется для определения области в форме круга для 2D объектов, то есть он используется для обнаружения столкновений с другими объектами и взаимодействия с ними на основе формы круга. Так как птичка имеет приближенную форму круга, ей данный компонент будет подходить, единственное, нужно изменить радиус под ее формы. Таким образом Radius = 0,23.

Теперь можно начать писать скрипт под названием «Player» для этого объекта. Все скрипты будут храниться в папке «Script». Для того, чтобы этот сценарий работал, его надо перетащить ко всем компонентам, только тогда он станет единым целым с объектом птички. Данный скрипт реализует прыжки игрока с гравитацией, анимацию спрайтов и обработку пользовательского ввода для управления игроком. 

Написанный скрипт подробно расписан в приложении.
%Будет расписан чуть попозже...


\chapter{ГЛАВА 3. ПРАКТИЧЕСКАЯ ЧАСТЬ С ТЕСТИРОВАНИЕМ}

\chapter*{Заключение}
\phantomsection\addcontentsline{toc}{chapter}{ЗАКЛЮЧЕНИЕ}

В целом, современные возможности создания игровых приложений позволяют разработчикам создавать увлекательные и качественные игры, которые могут быть доступны пользователям на разных платформах, не вызывая конфликтов и ограничений. Это открывает новые горизонты для игровой индустрии и предоставляет игрокам возможность наслаждаться играми, независимо от выбранной операционной системы или устройства.

Так, благодаря данной работе, было разработано игровое приложение «Flappy Bird» с помощью интегрированной среды разработки Unity и встроенного в него языка программирования С\#. С помощью упрощенного интерфейса разработки среды и использованию различных методов языка цель по созданию кроссплатформенной игры с графическим интерфейсом была достигнута.

\newpage
\phantomsection\addcontentsline{toc}{chapter}{СПИСОК ИСПОЛЬЗОВАННОЙ ЛИТЕРАТУРЫ}
\printbibliography[title={Список использованной литературы}]

\appendix
\newpage
\chapter*{\raggedleft\label{appendix1}Приложение}
\phantomsection\addcontentsline{toc}{chapter}{ПРИЛОЖЕНИЕ}
%\section*{\centering\label{code:appendix}Текст программы}

\begin{center}
\label{code:appendix}Текст программы
\end{center}


\end{document}
